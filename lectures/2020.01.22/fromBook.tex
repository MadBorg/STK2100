\documentclass[]{article}
\usepackage{lmodern}
\usepackage{amssymb,amsmath}
\usepackage{ifxetex,ifluatex}
\usepackage{fixltx2e} % provides \textsubscript
\ifnum 0\ifxetex 1\fi\ifluatex 1\fi=0 % if pdftex
  \usepackage[T1]{fontenc}
  \usepackage[utf8]{inputenc}
\else % if luatex or xelatex
  \ifxetex
    \usepackage{mathspec}
  \else
    \usepackage{fontspec}
  \fi
  \defaultfontfeatures{Ligatures=TeX,Scale=MatchLowercase}
\fi
% use upquote if available, for straight quotes in verbatim environments
\IfFileExists{upquote.sty}{\usepackage{upquote}}{}
% use microtype if available
\IfFileExists{microtype.sty}{%
\usepackage[]{microtype}
\UseMicrotypeSet[protrusion]{basicmath} % disable protrusion for tt fonts
}{}
\PassOptionsToPackage{hyphens}{url} % url is loaded by hyperref
\usepackage[unicode=true]{hyperref}
\hypersetup{
            pdftitle={R Notebook},
            pdfborder={0 0 0},
            breaklinks=true}
\urlstyle{same}  % don't use monospace font for urls
\usepackage[margin=1in]{geometry}
\usepackage{color}
\usepackage{fancyvrb}
\newcommand{\VerbBar}{|}
\newcommand{\VERB}{\Verb[commandchars=\\\{\}]}
\DefineVerbatimEnvironment{Highlighting}{Verbatim}{commandchars=\\\{\}}
% Add ',fontsize=\small' for more characters per line
\usepackage{framed}
\definecolor{shadecolor}{RGB}{248,248,248}
\newenvironment{Shaded}{\begin{snugshade}}{\end{snugshade}}
\newcommand{\KeywordTok}[1]{\textcolor[rgb]{0.13,0.29,0.53}{\textbf{#1}}}
\newcommand{\DataTypeTok}[1]{\textcolor[rgb]{0.13,0.29,0.53}{#1}}
\newcommand{\DecValTok}[1]{\textcolor[rgb]{0.00,0.00,0.81}{#1}}
\newcommand{\BaseNTok}[1]{\textcolor[rgb]{0.00,0.00,0.81}{#1}}
\newcommand{\FloatTok}[1]{\textcolor[rgb]{0.00,0.00,0.81}{#1}}
\newcommand{\ConstantTok}[1]{\textcolor[rgb]{0.00,0.00,0.00}{#1}}
\newcommand{\CharTok}[1]{\textcolor[rgb]{0.31,0.60,0.02}{#1}}
\newcommand{\SpecialCharTok}[1]{\textcolor[rgb]{0.00,0.00,0.00}{#1}}
\newcommand{\StringTok}[1]{\textcolor[rgb]{0.31,0.60,0.02}{#1}}
\newcommand{\VerbatimStringTok}[1]{\textcolor[rgb]{0.31,0.60,0.02}{#1}}
\newcommand{\SpecialStringTok}[1]{\textcolor[rgb]{0.31,0.60,0.02}{#1}}
\newcommand{\ImportTok}[1]{#1}
\newcommand{\CommentTok}[1]{\textcolor[rgb]{0.56,0.35,0.01}{\textit{#1}}}
\newcommand{\DocumentationTok}[1]{\textcolor[rgb]{0.56,0.35,0.01}{\textbf{\textit{#1}}}}
\newcommand{\AnnotationTok}[1]{\textcolor[rgb]{0.56,0.35,0.01}{\textbf{\textit{#1}}}}
\newcommand{\CommentVarTok}[1]{\textcolor[rgb]{0.56,0.35,0.01}{\textbf{\textit{#1}}}}
\newcommand{\OtherTok}[1]{\textcolor[rgb]{0.56,0.35,0.01}{#1}}
\newcommand{\FunctionTok}[1]{\textcolor[rgb]{0.00,0.00,0.00}{#1}}
\newcommand{\VariableTok}[1]{\textcolor[rgb]{0.00,0.00,0.00}{#1}}
\newcommand{\ControlFlowTok}[1]{\textcolor[rgb]{0.13,0.29,0.53}{\textbf{#1}}}
\newcommand{\OperatorTok}[1]{\textcolor[rgb]{0.81,0.36,0.00}{\textbf{#1}}}
\newcommand{\BuiltInTok}[1]{#1}
\newcommand{\ExtensionTok}[1]{#1}
\newcommand{\PreprocessorTok}[1]{\textcolor[rgb]{0.56,0.35,0.01}{\textit{#1}}}
\newcommand{\AttributeTok}[1]{\textcolor[rgb]{0.77,0.63,0.00}{#1}}
\newcommand{\RegionMarkerTok}[1]{#1}
\newcommand{\InformationTok}[1]{\textcolor[rgb]{0.56,0.35,0.01}{\textbf{\textit{#1}}}}
\newcommand{\WarningTok}[1]{\textcolor[rgb]{0.56,0.35,0.01}{\textbf{\textit{#1}}}}
\newcommand{\AlertTok}[1]{\textcolor[rgb]{0.94,0.16,0.16}{#1}}
\newcommand{\ErrorTok}[1]{\textcolor[rgb]{0.64,0.00,0.00}{\textbf{#1}}}
\newcommand{\NormalTok}[1]{#1}
\usepackage{graphicx,grffile}
\makeatletter
\def\maxwidth{\ifdim\Gin@nat@width>\linewidth\linewidth\else\Gin@nat@width\fi}
\def\maxheight{\ifdim\Gin@nat@height>\textheight\textheight\else\Gin@nat@height\fi}
\makeatother
% Scale images if necessary, so that they will not overflow the page
% margins by default, and it is still possible to overwrite the defaults
% using explicit options in \includegraphics[width, height, ...]{}
\setkeys{Gin}{width=\maxwidth,height=\maxheight,keepaspectratio}
\IfFileExists{parskip.sty}{%
\usepackage{parskip}
}{% else
\setlength{\parindent}{0pt}
\setlength{\parskip}{6pt plus 2pt minus 1pt}
}
\setlength{\emergencystretch}{3em}  % prevent overfull lines
\providecommand{\tightlist}{%
  \setlength{\itemsep}{0pt}\setlength{\parskip}{0pt}}
\setcounter{secnumdepth}{0}
% Redefines (sub)paragraphs to behave more like sections
\ifx\paragraph\undefined\else
\let\oldparagraph\paragraph
\renewcommand{\paragraph}[1]{\oldparagraph{#1}\mbox{}}
\fi
\ifx\subparagraph\undefined\else
\let\oldsubparagraph\subparagraph
\renewcommand{\subparagraph}[1]{\oldsubparagraph{#1}\mbox{}}
\fi

% set default figure placement to htbp
\makeatletter
\def\fps@figure{htbp}
\makeatother


\title{R Notebook}
\author{}
\date{\vspace{-2.5em}}

\begin{document}
\maketitle

\section{2.1}\label{section}

\subsection{2.1.1 Basic Concepts}\label{basic-concepts}

Simple practical problem:

\begin{verbatim}
Identify a relationship that allows us to predict the consumption of fuel, or equivalently, the distance covered per unit of fuel as a function of certain characteristics of a car. We consider data from 203 models of cars in circulation in 1985 in the US, but produced elsewhere. We have 27 different characteristics for each car, four of which are city distance (km/L), engine size (L), number of cylinders, and curb weight (kg). The data is marked for diesel and gasoline.
\end{verbatim}

Some of the data are numerical:

\begin{itemize}
\tightlist
\item
  Quantitative and continuous: City distance, engine size, and curb
  weight (kg)
\item
  Quantitative and discrete: number of cylinders
\end{itemize}

Matrix of scaterplots of some variables of car data, stratified by fuel
type.

\begin{Shaded}
\begin{Highlighting}[]
\NormalTok{auto <-}\StringTok{ }\KeywordTok{read.table}\NormalTok{(}\StringTok{"http://azzalini.stat.unipd.it/Book-DM/auto.dat"}\NormalTok{, }\DataTypeTok{header =} \OtherTok{TRUE}\NormalTok{)}
\KeywordTok{attach}\NormalTok{(auto)}
\CommentTok{#summary(auto)}
\CommentTok{# Sample size}
\NormalTok{n =}\StringTok{ }\KeywordTok{nrow}\NormalTok{(auto)}

\CommentTok{# Create dummy variable for fuel: disel = False, gasoline = TRUE}
\NormalTok{d =}\StringTok{ }\NormalTok{fuel }\OperatorTok{==}\StringTok{ "gas"}


\CommentTok{# Scater-plot matrix}
\KeywordTok{pairs}\NormalTok{(auto[ , }\KeywordTok{c}\NormalTok{(}\StringTok{"city.distance"}\NormalTok{, }\StringTok{"engine.size"}\NormalTok{,}\StringTok{"n.cylinders"}\NormalTok{,}\StringTok{"curb.weight"}\NormalTok{)],     }
      \DataTypeTok{labels =} \KeywordTok{c}\NormalTok{(}\StringTok{"City}\CharTok{\textbackslash{}n}\StringTok{distance"}\NormalTok{, }\StringTok{"Engine}\CharTok{\textbackslash{}n}\StringTok{size"}\NormalTok{,}\StringTok{"Number of}\CharTok{\textbackslash{}n}\StringTok{cylinders"}\NormalTok{, }\StringTok{"Curb}\CharTok{\textbackslash{}n}\StringTok{weight"}\NormalTok{),}
      \DataTypeTok{col =} \KeywordTok{ifelse}\NormalTok{(d, }\StringTok{'blue'}\NormalTok{, }\StringTok{'red'}\NormalTok{), }\DataTypeTok{pch =} \KeywordTok{ifelse}\NormalTok{(d, }\DecValTok{1}\NormalTok{, }\DecValTok{2}\NormalTok{), }\CommentTok{# 1 is a circle, 2 is a triangle}
      \DataTypeTok{cex =} \DecValTok{10}\OperatorTok{/}\KeywordTok{sqrt}\NormalTok{(n))}
\end{Highlighting}
\end{Shaded}

\includegraphics{fromBook_files/figure-latex/unnamed-chunk-1-1.pdf}

Some are qualitative:

\begin{itemize}
\tightlist
\item
  fuel type: diesel and gasoline
\end{itemize}

\textbf{We will in first phase consider only two explanatory variables:}

\begin{itemize}
\tightlist
\item
  Engine size, fuel type
\end{itemize}

To study the relationship between quantitative variables we usually make
a grafic representation.

\begin{Shaded}
\begin{Highlighting}[]
\NormalTok{### figure 2.2 }\AlertTok{###}
\KeywordTok{plot}\NormalTok{(engine.size, city.distance, }\DataTypeTok{type =} \StringTok{"n"}\NormalTok{, }\DataTypeTok{ylab =} \StringTok{"City distance (km/L)"}\NormalTok{,}
     \DataTypeTok{xlab =} \StringTok{"Engine size (L)"}\NormalTok{, }\DataTypeTok{xlim =} \KeywordTok{c}\NormalTok{(}\DecValTok{1}\NormalTok{, }\FloatTok{5.5}\NormalTok{))}
\KeywordTok{points}\NormalTok{(engine.size[d], city.distance[d], }\DataTypeTok{col =} \DecValTok{4}\NormalTok{, }\DataTypeTok{pch =} \DecValTok{1}\NormalTok{)}
\KeywordTok{points}\NormalTok{(engine.size[}\OperatorTok{!}\NormalTok{d], city.distance[}\OperatorTok{!}\NormalTok{d], }\DataTypeTok{col =} \DecValTok{2}\NormalTok{, }\DataTypeTok{pch =} \DecValTok{2}\NormalTok{)}
\KeywordTok{legend}\NormalTok{(}\StringTok{'topright'}\NormalTok{, }\DataTypeTok{pch =} \KeywordTok{c}\NormalTok{(}\DecValTok{1}\NormalTok{, }\DecValTok{2}\NormalTok{), }\DataTypeTok{col =} \KeywordTok{c}\NormalTok{(}\DecValTok{4}\NormalTok{, }\DecValTok{2}\NormalTok{),}
       \DataTypeTok{legend =} \KeywordTok{c}\NormalTok{(}\StringTok{"Gasoline  "}\NormalTok{,}\StringTok{"Diesel"}\NormalTok{))}
\end{Highlighting}
\end{Shaded}

\includegraphics{fromBook_files/figure-latex/unnamed-chunk-2-1.pdf}

We first suggest a simple linear regression line:
\(y = \beta_{0} + \beta_{1} x + \epsilon\), where y represents city
distance, x fuel type, and \(\epsilon\) is a nonobservable random
`error', with expected value 0 and variance \(\sigma^2\). For simplicity
we consider no correlation between error terms and y. We need to find a
setimator for the unknown variales \(\beta_{0}\) and \(\beta_{1}\). To
do so wee need to user the method of least squares. Wich means the
finding the min of the function:

\center \(B(\beta)=\Sigma_{i=1}^n \{y_{i} -f(x_{i};\beta) \}^2 = ||y-f(x;\beta)||^2\)
\center

The last expressison is showing the matrix notation for representing
\(y = (y1,...,y_{n})^T; (f(x_{1};\beta),...,f(x_{m};\beta))^T;\).

\textbf{But from what we can see in the plot of `city distance against
engine size', a linear model might not be the best.} Therefor we might
consider a polynomial form, since it is easy and quick. The polynom will
look something like:
\center \(f(x;\beta) = \beta_{0} + \beta_{1} x + ... + \beta_{p-1} x^{p-1}\)
\center

Because of the simplicity if a polynomial we can wirte f in matrix form
with low effort, \center \(f(x;\beta)= X\beta\) \center
where x is an n x p matrix, called the design matrix, defined by
\cnter \(X=(1,x,...,x^{p-1})\) \center
where x is the vector of the observations of the explanatory variable,
and the varius columns of X contains powers of order from 0 to p-1 of
elements of x.

On the matrix form the solution to the minimaztion problem is
\center \(\hat{\beta} = (X^{T}X)^{-1}X^{T}y\)

\subsection{2.3 Multivariable Responses}\label{multivariable-responses}

In many cases we would like more than one responsvariable. In the car
example we for example would like higway and city distance to be out
responsvariables.

If we create q response variables, we can construct a matrix Y such that
the columns that contain the q variables. In pur example we have q = 2,
and then

\center \(Y = ((CityDistance), (HighwayDistance))\). \center

If we create q models of linear regression, each of the type seen when
we only had one response, using the same regression matrix X for each,
we obtain

\center \(Y = XB + E\) \center \hfill    (2.20)

where B us the matrix formed of \textbf{q} columns of dimention
\textbf{p}, each representing the regression parameters for the
corresponding column of Y, and matrix R is made up of error terms. Here
aswell each of its columns reffers to the corresponding column of Y,
with condition that

\center \(var[\tilde{E}_{i}] = \Sigma\) \center

where \(\tilde{E}_{i}^{T}\) represent the i'th row E, for i = 1,
\ldots{} , n, and \(\Sigma\) is a variance matrix of dimention q x q
independent of i, wich expresses the correlation structure between the
error components and therefor also between the response variables.

\end{document}
